\documentclass[13pt,ignorenonframetext,]{beamer}
%% My Adds
\usepackage[round]{natbib}
\usepackage[]{hyperref}
\renewcommand*{\bibfont}{\footnotesize} % bib font size
\usepackage{color}
\newcommand{\red}[1]{\textcolor{red}{#1}} % text color \red{foo}
\newcommand{\blue}[1]{\textcolor{blue}{#1}}

%%%%%
% \setbeamertemplate{caption}[numbered]
% \setbeamertemplate{caption label separator}{: }
% \setbeamercolor{caption name}{fg=normal text.fg}

\setbeamertemplate{caption}{% To disable figure capution
\begin{beamercolorbox}[wd=.5\paperwidth, sep=.2ex]{block body}\insertcaption%
\end{beamercolorbox}%
}% https://tex.stackexchange.com/a/82460


\beamertemplatenavigationsymbolsempty
\usepackage{lmodern}
\usepackage{amssymb,amsmath}
\usepackage{ifxetex,ifluatex}
\usepackage{fixltx2e} % provides \textsubscript
\ifnum 0\ifxetex 1\fi\ifluatex 1\fi=0 % if pdftex
  \usepackage[T1]{fontenc}
  \usepackage[utf8]{inputenc}
\else % if luatex or xelatex
  \ifxetex
    \usepackage{mathspec}
  \else
    \usepackage{fontspec}
  \fi
  \defaultfontfeatures{Ligatures=TeX,Scale=MatchLowercase}
\fi
% use upquote if available, for straight quotes in verbatim environments
\IfFileExists{upquote.sty}{\usepackage{upquote}}{}
% use microtype if available
\IfFileExists{microtype.sty}{%
\usepackage{microtype}
\UseMicrotypeSet[protrusion]{basicmath} % disable protrusion for tt fonts
}{}
\newif\ifbibliography
\usepackage{natbib}
\bibliographystyle{plainnat}
\hypersetup{
            pdftitle={Slides Title},
            pdfauthor={Author Name Affiliation},
            pdfborder={0 0 0},
            breaklinks=true}
\urlstyle{same}  % don't use monospace font for urls
\usepackage{longtable,booktabs}
\usepackage{caption}
% These lines are needed to make table captions work with longtable:
\makeatletter
\def\fnum@table{\tablename~\thetable}
\makeatother

% Prevent slide breaks in the middle of a paragraph:
\widowpenalties 1 10000
\raggedbottom

\AtBeginPart{
  \let\insertpartnumber\relax
  \let\partname\relax
  \frame{\partpage}
}
\AtBeginSection{
  \ifbibliography
  \else
    \let\insertsectionnumber\relax
    \let\sectionname\relax
    \frame{\sectionpage}
  \fi
}
\AtBeginSubsection{
  \let\insertsubsectionnumber\relax
  \let\subsectionname\relax
  \frame{\subsectionpage}
}

\setlength{\parindent}{0pt}
\setlength{\parskip}{6pt plus 2pt minus 1pt}
\setlength{\emergencystretch}{3em}  % prevent overfull lines
\providecommand{\tightlist}{%
  \setlength{\itemsep}{0pt}\setlength{\parskip}{0pt}}
\setcounter{secnumdepth}{0}
\usepackage{natbib}

\title{Slides Title}
\author{Author Name\\
Affiliation}
\date{Feb 2019}

\begin{document}
\frame{\titlepage}

\begin{frame}

\end{frame}

\begin{frame}{SLIDE 1: bullets example}

\begin{block}{This is a bullet}

Some text

\end{block}

\begin{block}{This is another bullet}

Some more text

\end{block}

\end{frame}

\begin{frame}{SLIDE 2: reference example}

\begin{block}{How to cite}

For example, WordNet18 is an old dataset \citep{miller1995wordnet}.

\end{block}

\end{frame}

\begin{frame}{SLIDE 3: table exmaple}

\tiny

\begin{longtable}[]{@{}ccc@{}}
\caption{Example table with reference
\citep{miller1995wordnet}}\tabularnewline
\toprule
head & relation & tail\tabularnewline
\midrule
\endfirsthead
\toprule
head & relation & tail\tabularnewline
\midrule
\endhead
\_\_radical\_NN\_1 & \_part\_of & \_\_molecule\_NN\_1\tabularnewline
\_\_physics\_NN\_1 & \_member\_of\_domain\_topic &
\_\_molecule\_NN\_1\tabularnewline
\_\_molecule\_NN\_1 & \_has\_part & \_\_atom\_NN\_1\tabularnewline
\bottomrule
\end{longtable}

\end{frame}

\begin{frame}{SLIDE 4: Math example}

\begin{block}{How to add math}

In word2vec's skip-gram model, the goal is to maximize the sum
log-likelihood given all training vocabulary as target:

\[
\sum_{t=1}^{T}\sum_{c\in C_t} \log p(w_c | w_t)
\]

Where:

\[
p(w_c | w_t) = \frac{exp(w_t^T . w_c)}{\sum_{w_i \in V}{exp(w_t^T w_i)}}
\]

\end{block}

\end{frame}

\renewcommand\refname{References}
\begin{frame}[allowframebreaks]{References}
\bibliographytrue
\bibliography{/Users/Aziz/Dropbox/thesis/myref.bib}
\end{frame}

\end{document}
